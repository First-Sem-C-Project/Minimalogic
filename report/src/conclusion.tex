\documentclass[report]{subfiles}
\begin{document}
    \chapter{CONCLUSION}
    	\section{Experience}
	When we started working on this project, we were sure it would be an interesting experience, but what we did not anticipate was the sheer number of challenges that would come along. Challenges that proved the knowledge gained from our course to be insufficient for a project of this level. Algorithms that were unfamiliar to us had to be used for various parts of the program. But, along the way, we learned that these challenges were part of the learning experience. All in all, this was a very helpful project that introduced us to different paradigms of programming.
As futile as it may be, we've tried to make a list of what we learned by doing this project:
    \begin{enumerate}
        \item{Organizing the project and making the source code readable by using multiple files and using a consistent naming convention.}
        \item{Collaborative work with VCS like Git and remote hosting services like GitHub.}
        \item{Using the official documentation of Libraries and the C language itself.}
        \item{Unique algorithms for things like drawing curves, collision detection, etc.}
        \item{Avoiding memory leaks and other vulnerabilities that are exposed in low level language such as C.}
        \item{Making sure the codes were methodical, operational and compatible.}
    \end{enumerate}
        \section{Overview of the project}
	A one sentence definition of this project could be : "An optimal use of C programming language, learnt throughout the semester, to create a Logics Simulator which will get the basic tasks done." 
	We can summarise the overall features of the project in these points:
    \begin{enumerate}
        \item{Features digital electronics components required to make the logic circuits.}
	\item{User friendly interactive approach to learning the basics.}
	\item{Allows the modification and analysing of the tasks as required by the user.}
	\item{Incorporates an engaging blueprint and layout.}
    \end{enumerate}
	\section{Possible Improvements}
	While this program is stable as far as we have tested it, it is nowhere near perfect. Many features were sratched off of the to-to list, some due to the lack of time and others because they were too complicated. Here is a list of improvements that can be made to this program.
    \begin{enumerate}
    	\item{Better graphics for components and wires. The wires look jagged right now which can be fixed with some anti-aliasing techniques.}
	\item{Selection of multiple components on the grid and the functionality to copy paste the selection.}
	\item{Zooming and panning to provide a larger canvas.}
	\item{More components. Currentlty, there are very few, basic components which makes drawing more complex circuits difficult.}
	\item{Letting users create custom components.}
    \end{enumerate}
\end{document}
