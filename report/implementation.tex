\documentclass[report]{subfiles}
\begin{document}
    \chapter{IMPLEMENTATION}
    The source code for this project can be accessed through the GitHub repository of the project:\\
    \url{https://github.com/Sem-Projects/logic-simulator-inator}
    \section{Brief explanation of the code}
    The source code for our project has been divided into 9 files which consists of 5 C source files and 4 header files. They have been briefly explained below.
    \subsection{component.c \& component.h}
    As the name suggests, these files contain all the necessary information about the components used in the program. The header file defines a structure named Component that encompasses the details about a component including its size, position, input source, number of inputs, input state(s), output state(s) and other information which is later used.
    The output of any component (except clock and state) depends on its input(s). To get the desired output for any component from its inputs, the source file defines different component specific functions. The working of these functions is pretty straight-forward as they follow the standard logic operations available in C. As for the clock, its output is generated based on the value of time variable, which changes as the program progresses, defined in program.c. The clock inverts its current state when time reaches a certain value. The output of state is inverted when the user clicks on it.
    \subsection{draw.h \& draw.c}
    These two files contain the variables and functions that are responsible for drawing all the elements that are visible on the screen such as Buttons, Components, and Wires. It also handles rendering text in the SDL window where necessary. The header file defines an enumeration of confirmation flags that are later used to ask the user for confirmation on certain operations.
    The standard rendering functions available in the SDL library are used in order to draw Buttons and Components. However, SDL does not offer the functionality to draw curves. So, a simple algorithm that approximates a cubic Bezier Curve is used to draw wires.
    As for displaying text, a character map consisting of all the ASCII characters is predefined when the program starts. The font used is robotoo.ttf. The character map is later used to display any text (ASCII based) on the screen.
    \subsection{interaction.h \& interaction.c}
    User interaction is an integral part of any program, even more so for programs that use both mouse and keyboard to take input. These two files are responsible for handling such interactions. The header file defines various structures that are necessary for the Undo/Redo functionalities.
    The source file defines different functions that determine what will happen when a certain button is pressed or when a component is placed on the grid. Since these functions handle interaction with the user, they are usually only called when an event occurs. An SDL event encompasses mouse clicks, keyboard presses, etc. Different functions are called for different events. This coordination is handled in the file program.c.
    \subsection{program.h \& program.c}
    To keep the main.c file clean, the main program loop is defined in this file. For this reason, it acts as the centerpiece of the program that coordinates the functions of all other files. To begin with, the header file defines macros for configuring the main window and different elements inside it. Also, the colors that are frequently used in the program are defined here.
    The source file can be vaguely divided into two parts: Initialization and Main Program Loop. The initialization part is responsible for setting up all the necessary elements needed for the program to function properly. This is a one-time process that occurs when the program is launched.
    The Main Program Loop, as the name suggests, is a loop that runs over and over until the user exits the program. Everything that the user does inside the program is handled in this section. During each loop, the program checks for events, performs necessary operations based on them, updates the elements on the screen if required, and redraws all of those elements.
    \subsection{main.c}
    As mentioned earlier, this file is kept as clean as possible by defining the main loop in program.c. Inside this file, the current working directory is changed to the folder containing the executable and font files, so that the font files are always found regardless of where the program is run from. This is done using the <direct.h> library.
    The main function calls functions for initialization, the main program loop, and finally closing the program.
\end{document}
