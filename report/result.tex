\documentclass[report.tex]{subfiles}
\begin{document}
\begin{large}
\chapter{RESULT}
	Upon running the program, the user will be greeted with a window as shown in Figure 1. 

Using this program is pretty straight-forward. The main menu is laid out on the left side of the canvas and all the available options are self-explanatory. However, this program is also capable of taking inputs from the keyboard. A list of valid keyboard commands is given below:
    \begin{itemize}
        \item{\keys{Shift} : When pressed during the placement of a new component, the component will be aligned with the grid.}
        \item{\keys{Delete}: This will delete a component that is either being hovered over or selected, the priority being the one that is hovered over.}
        \item{\keys{Equal} : If possible, the number of inputs of the component being placed on the grid will be increased by 1, maximum value being 5.}
        \item{\keys{Minus} : If possible, the number of inputs of the component being placed on the grid will be decreased by 1, minimum value being 2.}
        \item{\keys{Ctrl+Z}: Revert to the last change made on the grid (Undo).}
        \item{\keys{Ctrl+R}: Redo the last change made by Undo.}
        \item{\keys{Ctrl+S}: Save changes made to the file currently open, if no file is open then prompt the user to create a new file.}
        \item{\keys{Ctrl+O}: Creates a windows dialog box to open an existing file.}
    \end{itemize}

Some important snapshots from the program are shown below:

\end{large}
\end{document}
