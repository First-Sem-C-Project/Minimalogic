\documentclass[report.tex]{subfiles}
\begin{document}
\begin{large}
\section{INTRODUCTION}
    % what is the program
    Logic circuits are one of the core components of CPUs. These circuits allow manipulating binary data and carrying out various logical operations.
    Programs such as Proteus and Logisim can be used to simulate logic circuits.
    For our project, we decided to create a similar (albeit heavily simplified) logic simulator named \textbf{MinimaLogic}.
    \\\\
    \textbf{MinimaLogic} is a GUI based logic simulator that allows simulation of various logic gates and circuits.
    It allows users to create and simulate circuits ranging from a simple 1-bit adder to more complex circuits like 4-bit counters.
    In fact, one could create any circuit that fits within the available area using the components provided in the program.
    The program allows the user to interact via mouse and keyboard. To make the program user friendly, we tried to keep the controls as intuitive as possible.
    \\\\
    This program heavily relies on the Simple DirectMedia Layer(SDL2) Library for GUI elements as well as user interation/input. 
    Alongside SDL2, the program also utilizes SDL2\_ttf for font rendering as well as the Windows API for opening/saving files.
    Other standard headers that are available with all modern development environments have also been used.
    % describe a little bit about C
    \subsection{The C language}
    % describe a little about SDL
    % SDL program loop
    \subsection{Simple DirectMedia Layer(SDL2)}
    % describe program structure
    \subsection{Structure of Code}
    % build.bat baby
    \subsection{Building}
\end{large}
\end{document}
